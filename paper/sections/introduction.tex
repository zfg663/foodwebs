Ecosystems are networks of species in a habitat where the population of any species generally depends on the populations of all other species. 
Such complex inter-relations make the species vulnerable to changes in the environment \cite{larsen2005extinction, walther2010community, loreau2013biodiversity}.
[Something about stability and sustainability \cite{loreau2013biodiversity, chapin1996principles}.]
Recent research indicates that human activity is driving species extinct at a rate corresponding to that of a mass extinction \cite{ceballos2015accelerated, barnosky2011has}. Consequently, it is crucial to gain further understanding of the conditions, under which ecosystems are stable or susceptible to collapse.

Sustainable ecosystems are plentiful on Earth, yet stable ecosystems have so far shown to be difficult to construct due to their complexity \cite{may1972will,gardner1970connectance} [cite some Allesina papers].
When simplifying the Jacobian of food webs as random matrices, it has been shown under a range of scenarios, that stability is exceedingly unlikely.
Strong self-limitation of each species, mimicking cannibalism, could theoretically remedy this problem, but such strong self-limitation may not be realistic [references needed]. 

Tree-like food webs and ecosystems of tightly coupled predators and resource species have been found to be easier to stabilize \cite{haerter2016food,allesina2012stability}. However, actual ecosystems are often found to inhibit a more complex nature \cite{hall1993food}. 
% what do you mean by this?

As evidence is found both in the field and from simulations, it has been argued that specializing on one resource species \cite{drossel2004impact, de1995energetics} or omnivory in the ecosystems enhances stability \cite{emmerson2004weak} [But also that omnivory induces chaos...]. 
% needs more elaboration: what exactly did they do and find?

In this article, we build an evolutionary algorithm to simulate successive invasions, by allowing species to be added one by one with a randomly chosen set of interaction parameters. 
After each invasion attempt, the food web is allowed to equilibrate to a new steady state --- often leading to extinctions of the invader or any of the resident species.
We first compare simulated tree-like food webs to previous exact results. 
We then allow for invaders with two resource species, which can be any of the resident species or the basic nutrient source. 
We then compare this second class of ecosystem to the tree-like food webs, finding that [...]. % Briefly summarize results here

